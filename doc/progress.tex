\documentclass{article}
\usepackage[utf8]{inputenc}
\usepackage{amsmath}
\usepackage{graphicx}
\usepackage{amsfonts}
\usepackage{adjustbox}
\usepackage{listings}
\usepackage{color}
\usepackage{braket}
\definecolor{dkgreen}{rgb}{0,0.6,0}
\definecolor{gray}{rgb}{0.5,0.5,0.5
\definecolor{mauve}{rgb}{0.58,0,0.82}


\lstset{frame=tb,
  language=Python,
  aboveskip=3mm,
  belowskip=3mm,
  showstringspaces=false,
  columns=flexible,
  basicstyle={\small\ttfamily},
  numbers=none,
  numberstyle=\tiny\color{gray},
  keywordstyle=\color{blue},
  commentstyle=\color{dkgreen},
  stringstyle=\color{mauve},
  breaklines=true,
  breakatwhitespace=true,
  tabsize=3
}
\title{Analysis of the Quantum Advantages for Deep Hedging}

\author{Soham Deshpande\\ Srinandan Dasmahapatra}
\date{October 2024}

\begin{document}

\maketitle

\tableofcontents

\clearpage

\section{Abstract}
Deep hedging

\clearpage
\section{Hedging}
Reword:\\ 
------------------------------------\\
Classical derivative pricing models all rely on mathematical path generators,
however these are not designed to be realistic, instead they describe diffusion
in a risk-neutral measure, rather than the real-world measure, with variables
being kept to a minimal for easy computation. 
\\------------------------------------\\

Mathematical set up for hedging:

\clearpage
\section{Quantum Computing}

\section{Parameterised Circuits}

\subsection{Born Rule}
An essential part of modern quantum computing involves the existence of the Born
Rule. Born's measurement rule states that:
$$p(x) = |\langle x|\psi(\theta)|\rangle|^2$$
The state $|\psi(\theta)\rangle$ is generated by evolving the vacuum state $|0\rangle$
according to a Hamiltonian $H$ that is constructed from gates. Once combined, the 
gates form a paramaterised quantum circuit which is paramaterised by using the 
variables governing each gate, $\theta$. By tuning the values of $\theta_i$ one 
can allow for an evolution to any state that will serve as a solution to a given 
problem. \\ 
By taking the distribution associated to the state, $|\psi(\theta)\rangle$ we can 
treat the PQC as a generative model, which upon measuring in a given basis, will 
generate samples of a target distribution $\chi$. This model is paramaterised 
by $\theta$, which defines a quantum circuit $U(\theta)$ made up of a set of quantum 
gates such that:
$$|\psi(\theta)\rangle = U(\theta)|0\rangle^{\otimes n}$$
By measuring the circuit, we can obtain samples. Producing samples that emulate 
the target distribution involves minimising the parameters of the circuit $U(\theta)$, 
a process once convergence is reached, will generate accurate samples. 
\cite{bornmachine} 

\subsection{State Preparation}
State preperation is a core part of the hybrid quantum computing model. This 
concerns the most efficient methods to input classical data into a quantum system. 
Without the use of ancillary qubits, we can expect an exponentional circuit depth
to prepare an arbitrary quantum state. Using them we can reduce the depth to be 
subexponential scaling, with recent advancements reaching $\Theta(n)$ given $O(n^2)$
ancillary qubits.
\cite{stateprep1}
\cite{stateprep2}
We require state preperation to transfer the classical data, bits, onto the 
Hilbert space. This involves a function $\phi$ that maps the input vector to 
an output label. There are many encoding schemes, each of which aim to offer 
high information density and low error rates.
The main methods of state preperation include: basis, amplitude, angle encoding,
and QRAM. 

\subsubsection{Amplitude Encoding}
$$ |\psi_x\rangle = \sum_{i-1}^Nx_i|i\rangle$$

\subsubsection{Angle Encoding}

$$|x\rangle = \otimes^N_{i=1} cos(x_i)|0\rangle + sin(x_i)|1\rangle$$



\section{Quantum Architectures}
\subsection{Measurements}
\subsubsection{Complete Measurements}
In quantum mechanics, we can define measurement to be any process that probes a 
given quantum system to obtain information, we may also refer to this process as 
a measurement in the computational basis when focussing on quantum information 
science. Let's consider a quantum state 
$|\psi \rangle = \alpha_0|0\rangle+\alpha_1|1\rangle$ which gives us $|0\rangle$
with probability $|\alpha_0|^2$ and $|1\rangle$ with probability $|\alpha_1|^2$. 
If measured in the standard basis we would expect the outcome to be $|k\rangle$
(for k = 0,1)with a given probability. This outcome would result in the output 
state of the measurement gate to also be $|k\rangle$, resulting in the original 
state $|\psi\rangle$ to be irreversibly lost. We can refer to the process the state 
undergoes as a collapse of state. 


\subsection{Weak Measurements}

Consider an arbitrary system observable A. Assume a probe $[\hat{q}.\hat{p}] = i$
in a minimum uncertainty state, defined by $\sigma^{in}_p$,$\bar{p}^{in}$, and 
$\bar{q}^{in}=0$. We can then assume a Von Neumann type interaction between the 
observable of interest and the position of the particle, consequence being the 
probe receiving a momentum kick when it interacts with the system and that the 
change in the momentum is exactly equal to the observable to be measured, $\hat{A}$. 
Mathematically, we can define $\hat{A}(p^t) = p^t-p^{in}$ to be the estimate we 
get for $A$ from measuring probe $\bar{p}^t$. Then for intial system state $|\psi^{in}\rangle$, 
$E[\psi_{in}|A(p^t)|\psi^{in}]=\langle|\hat{A}|\psi^{in}\rangle$.\\
Now consider a final projective measurement on the system too, considering the 
sub-enemble where the system is found in state $|\phi^t\rangle$. Then we can consider 
the post-selected average $E[A(p^t)|\psi^{in},\phi^{t}]$.
In the weak measurement limit, $\sigma_p\rightarrow\infty$ we get the expectation value for the estimate 
of the observable A given by: 

$$E[A(p^t)|\psi^{in},\phi^t]\rightarrow Re\frac{\langle\phi^y|\hat{A}|\psi^{in}}{\langle\phi^t|\psi^t\rangle}$$
The amount of disturbance to any system operator due to the coupling of the probe is very small. 


Can we use weak measurement to sample the qcbm continuously?\\
look at the weak measurement protocol.
Different to the standard orthogonal projection. 
%https://github.com/Sjd-Hz/Weak-measurement-in-IBM-Qiskit/blob/main/How%20to%20implement%20weak%20measurement%20in%20IBM%20Qiskit.ipynb



\section{Quantum Resevoir Computing}


\clearpage
\section{Deep Hedging}


\clearpage 
\section{Bibliography}
To sort out later:\\
Born machine: https://link.springer.com/article/10.1007/s42484-022-00063-3
\\
Stateprep1: https://arxiv.org/pdf/2201.11495
\\
Stateprep2: https://arxiv.org/pdf/2108.06150
\end{document}
